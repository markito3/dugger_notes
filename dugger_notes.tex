\documentclass{article}

\usepackage{amsmath}
\usepackage{graphicx}

\begin{document}

Appendix A, first unnumbered equation is equivalent to 

$${d^3\sigma \over d\cos\theta d\phi E_\gamma}$$

which would make eq. (A1)

$$\Delta\sigma^{i,j,k} = \int^{E_i}_{E_{i-1}}\int^{\cos\theta_j}_{\cos\theta_{j-1}}\int^{\phi_k}_{\phi_{k-1}} {d^3\sigma \over d\cos\theta d\phi dE_\gamma}dE_\gamma d\cos\theta d\phi$$

and since

$${d\cos\theta \over d\theta} = -\sin\theta$$

we get

$$ d\Omega = d\cos\theta d\phi = -\sin\theta d\theta d\phi $$

(A1) becomes

$$\Delta\sigma^{i,j,k} = \int^{E_i}_{E_{i-1}}\int_{\theta_j}^{\theta_{j-1}}\int^{\phi_k}_{\phi_{k-1}} {d^3\sigma \over \sin\theta d\theta d\phi dE_\gamma} dE_\gamma \sin\theta d\theta d\phi$$

or

$$\Delta\sigma^{i,j,k} = \int^{E_i}_{E_{i-1}}\int_{\theta_j}^{\theta_{j-1}}\int^{\phi_k}_{\phi_{k-1}} {d^2\sigma \over d\Omega dE_\gamma} dE_\gamma \sin\theta d\theta d\phi$$

or

$$\Delta\sigma^{i,j,k} = \int^{E_i}_{E_{i-1}}\int_{\theta_j}^{\theta_{j-1}}\int^{\phi_k}_{\phi_{k-1}} {d^3\sigma \over d\theta d\phi dE_\gamma} dE_\gamma d\theta d\phi$$

as one might expect.

Probability of scattering is $p = \sigma/A$ where $\sigma$ is the total cross section for scattering, $A$ is the nominal area where scattering can occur. Let us call $N_t$ the number of target particles and $N_\gamma$ the number of beam particles. $N_t = \rho AL$ where $\rho$ is then target number volume density and $L$ is the length of the target. The number of scattering events $N_s$ is then
$$
N_s = N_\gamma N_t p = N_\gamma(\rho A L){\sigma\over A} = N_\gamma\rho L\sigma
$$

If we define the yield $Y$ as the number of detected scattering events, then $Y = \epsilon N_s$ where $\epsilon$ is the detection efficiency. So
$$
Y = N_\gamma\rho L\sigma\epsilon
$$
which recovers eq.~(A2). The yield on one bin is
$$
Y^{i,j,k} = N^i_\gamma\rho L\Delta\sigma^{i,j,k}\epsilon^{i,j,k}
$$

\begin{equation}
{d\sigma_\perp \over d\Omega} = {d\sigma_a \over d\Omega}[1+P_\perp\Sigma\cos 2\phi] \tag{A3} \\
\end{equation}
\begin{equation}
{d\sigma_\parallel \over d\Omega} = {d\sigma_a \over d\Omega}[1-P_\parallel\Sigma\cos 2\phi] \tag{A4} \\
\end{equation}

\includegraphics[width=\linewidth]{assymetry.png}

Define the yield per unit $\phi$ interval in the $i$th energy bin and $j$th theta bin
$$
Y^{i,j}(\phi) \approx { Y^{i,j,k} \over \Delta\phi_k }
$$

In eq.~(A5) all cross sections are measured for the same target. Flux and target density is eliminated by definition of cross section.



\end{document}

