\documentclass{article}

\begin{document}

Appendix A, first unnumbered equation is equivalent to 

$${d^3\sigma \over d\cos\theta d\phi E_\gamma}$$

which would make eq. (A1)

$$\Delta\sigma^{i,j,k} = \int^{E_i}_{E_{i-1}}\int^{\cos\theta_i}_{\cos\theta_{i-1}}\int^{\phi_i}_{\phi_{i-1}} {d^3\sigma \over d\cos\theta d\phi dE_\gamma}dE_\gamma d\cos\theta d\phi$$

and since

$${d\cos\theta \over d\theta} = -\sin\theta$$

we get

$$ d\Omega = d\cos\theta d\phi = -\sin\theta d\theta d\phi $$

(A1) becomes

$$\Delta\sigma^{i,j,k} = \int^{E_i}_{E_{i-1}}\int^{\theta_i}_{\theta_{i-1}}\int^{\phi_i}_{\phi_{i-1}} {d^3\sigma \over \sin\theta d\theta d\phi E_\gamma} dE_\gamma \sin\theta d\theta d\phi$$

or

$$\Delta\sigma^{i,j,k} = \int^{E_i}_{E_{i-1}}\int^{\theta_i}_{\theta_{i-1}}\int^{\phi_i}_{\phi_{i-1}} {d^3\sigma \over d\theta d\phi E_\gamma} dE_\gamma d\theta d\phi$$

as one might expect.

In eq. (A5) all cross sections are measured for the same target. Flux and target density is eliminated by definition of cross section.

Probability of scattering is $p = \sigma/A$ where $\sigma$ is the total cross section for scattering, $A$ is the nominal area where scattering can occur. Let us call $N_t$ the number of target particles and $N_b$ the number of beam particles. $N_t = \rho AL$ where $\rho$ is then target number volume density and $L$ is the length of the target. The number of scattering events $N_s$ is then
$$
N_s = N_b N_t p = N_b(\rho A L){\sigma\over A} = N_b\rho L\sigma
$$

If we define the yield $Y$ as the number of detected scattering events, then $Y = \epsilon N_s$ where $\epsilon$ is the detection efficiency. So
$$
Y = N_b\rho L\sigma\epsilon
$$
which recovers eq.~(A2).

\end{document}

